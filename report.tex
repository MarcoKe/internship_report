\documentclass{article}
\usepackage[utf8]{inputenc}

\title{Internship Report: Neocortex stuff}
\author{Marco Kemmerling}
\date{\today}

\usepackage{natbib}
\usepackage{graphicx}
\usepackage{amssymb}
\usepackage{amsmath}
\usepackage{url}
\usepackage{array}

\begin{document}

\maketitle


\section{Neocortex} 
relatively recent addition to brains, makes up 80\% human brain \\
bla bla plasticity single learning principle in all of the cortex \newline \newline
Regular structure: The cortex is organised horizontally into six layers, and vertically into groups of cells synaptically linked across the horizontal layers \cite{Mountcastle1997}. \\ Six layers L1-L6. What is each layers function, properties, etc.? \newline \newline
Cortical column: It can be observed that neurons stacked on top of each others tend to be connected and have similar response properties. Hence, the cortex appears to be organised in a columnar way and these columns have been hypothesised to represent a basic functional unit for sensory processing or motor output. The majority of connections are between neurons in the same column, while only a minority is between columns.  \cite{goodhill2002cortical} \\ 
It should be noted that the term column is not always well defined. While columns in some regions have clear discrete boundaries, columns in other regions may be more difficult to distinguish from one another. The term column can further refer to structure at several different scales, e.g. mini columns and hyper columns.  \\ 
- Sensory input predominantly arrives from the thalamus in layer 4. \\ 

\subsection{The Role Of The Thalamus}
- relays sensory information to cortex (to relatively specific regions in the cortex, hence called specific nucleus) \\ 
- serves as a hub for feedback interaction between cortical regions. That is, most areas of the thalamus receive inputs predominantly from the cortex and project back to it in a relatively non-specific fashion, hence the term non specific-nucleus. 

\section{Artificial Neural Networks} 
- Simplified neurons etc. \\ 

\newpage


\bibliographystyle{plain}
\bibliography{refs}
\end{document}
